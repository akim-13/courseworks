\documentclass[12pt]{article}

% Packages
\usepackage{amsmath, amsthm, amssymb, amsfonts}
\usepackage[hidelinks]{hyperref}
\usepackage{enumitem}
\usepackage{geometry}
\usepackage{fancyhdr}

% Page Geometry
\geometry{a4paper, margin=1in, headheight=14.5pt}

% Header and Footer
\pagestyle{fancy}
\fancyhf{}
\lhead{Discrete Mathematics Problem Sheet}
\rhead{CM12004}
\cfoot{\thepage}

% Custom Commands
\newcommand{\p}[1]{\item[\textnormal{(#1)}]}
\newcommand{\q}{\hfill $\blacksquare$}

\newenvironment{ps}
{\begin{enumerate}[leftmargin=0em, itemindent=1.5em]}
{\end{enumerate}}

\begin{document}

\begin{center}
\begin{tabular}{|c|c|}
\hline
\textbf{Notation} & \textbf{Meaning}  \\
\hline

T & True  \\
F & False  \\
\(\oplus\) & XOR \\
\( \downarrow \) & NOR \\
\( \uparrow \) & NAND \\
\(\mathbb{E}\) & Set of all even integers \\
\(\mathbb{O}\) & Set of all odd integers \\
\( \blacksquare \) & Indicates the end of proof/solution \\
\( \overset{d}{\equiv} \) & Equivalent by definition \\

\hline
\end{tabular}
\end{center}

\section*{1.}
\begin{ps}

    \p{i} The formula \(P \rightarrow Q\) is \textbf{neither} a tautology nor identically
    false.
        \begin{center}
        \begin{tabular}{|c|c|c|c|c|}
        \hline
        $X$ & $Y$ & $P = X \vee Y$ & $Q = \neg (X \wedge Y)$ & $P \rightarrow Q$ \\
        \hline
        T & T & T & F & F \\
        T & F & T & T & T \\
        F & T & T & T & T \\
        F & F & F & T & T \\
        \hline
        \end{tabular}
        \end{center}

    \p{ii} The formula \(P \rightarrow Q\) is \textbf{neither} a tautology nor identically
    false.

        \begin{center}
        \begin{tabular}{|c|c|c|c|c|}
        \hline
        $X$ & $Y$ & $P = X \vee Y$ & $Q = \neg X \wedge \neg Y$ & $P \rightarrow Q$ \\
        \hline
        T & T & T & F & F \\
        T & F & T & F & F \\
        F & T & T & F & F \\
        F & F & F & T & T \\
        \hline
        \end{tabular}
        \end{center}

    \p{iii} The formula \(P \rightarrow Q\) is a \textbf{tautology}. Note that the second
    part of Q (\(\neg X \vee X\)) is also a tautology.

        \begin{center}
        \begin{tabular}{|c|c|c|c|c|}
        \hline
        $X$ & $Y$ & $P = X \rightarrow Y$ & $Q = (\neg X \vee Y) \wedge (\neg X \vee X)$ & $P \rightarrow Q$ \\
        \hline
        T & T & T & T & T \\
        T & F & F & F & T \\
        F & T & T & T & T \\
        F & F & T & T & T \\
        \hline
        \end{tabular}
        \end{center}

    \p{iv} The formula \(P \rightarrow Q\) is a \textbf{tautology}.

        \begin{center}
        \begin{tabular}{|c|c|c|c|c|}
        \hline
        $X$ & $Y$ & $P = X \rightarrow \neg Y$ & $Q = Y \rightarrow \neg X$ & $P \rightarrow Q$ \\
        \hline
        T & T & F & F & T \\
        T & F & T & T & T \\
        F & T & T & T & T \\
        F & F & T & T & T \\
        \hline
        \end{tabular}
        \end{center}

    \p{v} The formula \(P \rightarrow Q\) is a \textbf{tautology}.

        \begin{center}
        \begin{tabular}{|c|c|c|c|c|c|}
        \hline
        $X$ & $Y$ & $Z$ & $P = X \wedge (Y \vee Z)$ & $Q = (X \vee Y) \wedge (X \vee Z)$ & $P \rightarrow Q$ \\
        \hline
        T & T & T & T & T & T \\
        T & T & F & T & T & T \\
        T & F & T & T & T & T \\
        T & F & F & F & T & T \\
        F & T & T & F & T & T \\
        F & T & F & F & F & T \\
        F & F & T & F & F & T \\
        F & F & F & F & F & T \\
        \hline
        \end{tabular}
        \end{center}

    \p{vi} The formula \(P \rightarrow Q\) is \textbf{neither} a tautology nor identically
    false.

        \begin{center}
        \begin{tabular}{|c|c|c|c|c|}
        \hline
        $X$ & $Y$ & $P = X \rightarrow Y$ & $Q = \neg X \rightarrow \neg Y$ & $P \rightarrow Q$ \\
        \hline
        T & T & T & T & T \\
        T & F & F & T & T \\
        F & T & T & F & F \\
        F & F & T & T & T \\
        \hline
        \end{tabular}
        \end{center}

    \p{vii} The formula \(P \rightarrow Q\) is \textbf{neither} a tautology nor
    identically false.

        \begin{center}
        \begin{tabular}{|c|c|c|c|c|}
        \hline
        $X$ & $Y$ & $P = X \rightarrow Y$ & $Q = \neg (Y \rightarrow X)$ & $P \rightarrow Q$ \\
        \hline
        T & T & T & F & F \\
        T & F & F & F & T \\
        F & T & T & T & T \\
        F & F & T & F & F \\
        \hline
        \end{tabular}
        \end{center}

    \p{viii} The formula \(P \rightarrow Q\) is a \textbf{tautology}.

        \begin{center}
        \begin{tabular}{|c|c|c|c|c|c|}
        \hline
        $X$ & $Y$ & $Z$ & $P = (Y \rightarrow Z) \wedge (X \rightarrow Y) $ & $Q = X \rightarrow Z$ & $P \rightarrow Q$ \\
        \hline
        T & T & T & T & F & T \\
        T & T & F & F & F & T \\
        T & F & T & F & T & T \\
        T & F & F & F & F & T \\
        F & T & T & T & T & T \\
        F & T & F & F & T & T \\
        F & F & T & T & T & T \\
        F & F & F & T & T & T \\
        \hline
        \end{tabular}
        \end{center}

\end{ps}


\section*{2.}
\begin{ps}

    \p{a} There exist \textbf{16} different binary logical connectives. There are
    4 possible combinations of T and F that a binary connective can take in. Each
    combination can result in 2 outcomes (T or F), therefore \(2^4 = 16\).

    \p{b} This section will make use of the following identities:

        \begin{enumerate}[label=\arabic*.]
            \item \( X \wedge Y \equiv \neg(\neg X \vee \neg Y) \)
            \item \( X \vee Y \equiv \neg(\neg X \wedge \neg Y) \)
            \item \( X \rightarrow Y \equiv \neg X \vee Y \)
        \end{enumerate}

    \p{i} Consider the definition of XOR: 
        \begin{equation}
            \label{XOR}
            X \oplus Y \equiv (X \vee Y) \wedge \neg(X \wedge Y)
        \end{equation}

        Let us transform equation \eqref{XOR} as follows:
        \begin{align*}
            (X \vee Y) \wedge \neg(X \wedge Y) &\equiv (X \vee Y) \wedge (\neg X \vee \neg Y) \\
                                               &\equiv \neg(\neg X \wedge \neg Y) \wedge \neg (X
                                               \wedge Y)
        \end{align*} \q

    \p{ii} Let us transform equation \eqref{XOR} as follows: 
        \begin{align}
            \label{b ii}
            (X \vee Y) \wedge \neg(X \wedge Y) &\equiv (X \vee Y) \wedge (\neg X \vee \neg Y) \notag \\
                                               &\equiv \neg (\neg (X \vee Y) \vee \neg (\neg X
                                               \vee Y ))
        \end{align} \q

    \p{iii} Let us transform equation \eqref{b ii} as follows: 
        \begin{align*}
            \neg (\neg (X \vee Y) \vee \neg (\neg X \vee Y )) &\equiv \neg(\neg(\neg X \rightarrow
            Y) \vee \neg(X \rightarrow \neg Y)) \\
            &\equiv \neg ((\neg X \rightarrow Y) \rightarrow \neg (X \rightarrow \neg Y))
        \end{align*} \q

    \p{c} By definition of NAND: 
        \[
            X \wedge Y \equiv \neg (X \uparrow Y)
        \]   

        Consider the truth table for NAND:

        \begin{center}
        \begin{tabular}{|c|c|c|}
        \hline
        $X$ & $Y$ & $X \uparrow Y$ \\
        \hline
        T & T & F \\
        T & F & T \\
        F & T & T \\
        F & F & T \\
        \hline
        \end{tabular}
        \end{center}

        Notice that \( X \uparrow X \equiv \neg X \). Hence: 
        \begin{align*}
            &\neg (X \uparrow Y) \equiv (X \uparrow Y) \uparrow (X \uparrow Y) \Leftrightarrow \\
            &X \wedge Y \equiv (X \uparrow Y) \uparrow (X \uparrow Y)
        \end{align*}
        Similarly, using the second and third identities, the remaining two connectives can be
        expressed as follows: 
        \begin{align*}
            X \vee Y &\equiv \neg (\neg X \wedge \neg Y) \\
                     &\equiv \neg ((X \uparrow X) \wedge (Y \uparrow Y)) \\
                     &\overset{d}{\equiv} (X \uparrow X) \uparrow (Y \uparrow Y) \\
            X \rightarrow Y &\equiv \neg X \vee Y \\
                            &\equiv \neg (X \wedge \neg Y) \\
                            &\overset{d}{\equiv} X \uparrow \neg Y \\
                            &\equiv X \uparrow (Y \uparrow Y)
        \end{align*} \q

    \p{d} By definition of NOR: 
        \[
            X \vee Y \equiv \neg (X \downarrow Y)
        \]   

        Consider the truth table for NOR:

        \begin{center}
        \begin{tabular}{|c|c|c|}
        \hline
        $X$ & $Y$ & $X \downarrow Y$ \\
        \hline
        T & T & F \\
        T & F & F \\
        F & T & F \\
        F & F & T \\
        \hline
        \end{tabular}
        \end{center}

        Notice that \( X \downarrow X \equiv \neg X \). Hence: 
        \begin{align*}
            &\neg (X \downarrow Y) \equiv (X \downarrow Y) \downarrow (X \downarrow Y) \Leftrightarrow \\
            &X \vee Y \equiv (X \downarrow Y) \downarrow (X \downarrow Y)
        \end{align*}

        Similarly, using the second and third identities, the remaining two connectives can be
        expressed as follows: 
        \begin{align*}
            X \wedge Y &\equiv \neg (\neg X \vee \neg Y) \\
                     &\overset{d}{\equiv} \neg X \downarrow \neg Y \\
                     &\equiv (X \downarrow X) \downarrow (Y \downarrow Y) \\
            X \rightarrow Y &\equiv \neg X \vee Y \\
                            &\equiv \neg \neg (\neg X \vee Y) \\
                            &\overset{d}{\equiv} \neg (\neg X \downarrow Y)\\
                            &\equiv ((X \downarrow X) \downarrow Y) \downarrow ((X \downarrow X)
                            \downarrow Y)
        \end{align*} \q

\end{ps}


\section*{3.}
\begin{ps}

    \p{a}
    \p{i} Let \( y = x^2 \), then the given predicate transforms into a tautology:
        \[ 
            x^2 < x^2 + 1 \implies 0< 1 
        \]
        Hence, the given statement is \textbf{true}.

    \p{ii} The given statement is \textbf{false}. Consider a counterexample. Let \( y = -1 \), then: 
    \[
        x^2 < -1 + 1 \implies x^2 < 0
    \]   
    Since no such integer \( x \) exists, the above statement is false.
    
    \p{iii} The given statement is \textbf{false}. Suppose such \( y \) exists. Let \( x = y + 2 \),
    then: 
    \begin{align*}
        (y + 2)^2 &< y + 1 \\
        y^2 + 4y + 4 &< y + 1 \\
        y^2 + 3y + 3 &< 0 \\
    \end{align*}
    However, the quadratic \( y^2 + 3y + 3 \) is always positive for all integer numbers \( y \),
    therefore the assumption leads to a contradiction, and no such \( y \) exists.

    \p{iv} Let \( y = 2x \), then:
    \begin{align*}
        (x < 2x) &\rightarrow (x^2 < 4x^2) \\
        (x > 0) &\rightarrow (x^2 > 0)
    \end{align*}
    Since the above statement is a tautology, the given statement is \textbf{true}.

    \p{b}
    \p{i} Let \( x = -1, y = -1 \), then: 
    \[ 
        (-1)^2 < -1 + 1 \Leftrightarrow 1 < 0 \text{ is false.}
    \]\q

    \p{ii} Let \( x = 2, y = 2 \), then:
    \[
        2^2 < 4 + 1 \Leftrightarrow 4 < 5 \text{ is true.} 
    \]\q

    \p{iii} Let \( y = -1 \), then:
    \[
        x^2 < -1 + 1 \Leftrightarrow x^2 < 0 \text{ is identically false.}
    \]\q

    \p{iv} Let \( x = 0 \), then:
    \[
        (y > 0) \rightarrow (y^2 > 0) \text{ is a tautology.}
    \]\q

\end{ps}


\section*{4.}

\begin{ps}
    \p{a} This section will make use of the following theorem: \( |2^A| = 2^{|A|} \).
    \begin{align*}
        & |2^\emptyset| = 2^{|\emptyset|} = 2^0 = 1 \\
        & |2^{\{0\}}| = 2^1 = 2 \\
        & |2^{\{0\} \cup \{1\}}| = |2^{\{0,1\}}| = 2^2 = 4 \\
        & |2^{\{0\} \cap \{1\}}| = |2^\emptyset| = 1 \\
        & |2^{\{\emptyset, 0, 1\}}| = 2^3 = 8 \\
        &\left|2^{2^{2^{\{0,1\}}}}\right| = 2^{\left|2^{2^{\{0, 1\}}}\right|} 
        = 2^{2^{\left| 2^{\{0, 1\}}\right|}} = 2^{2^4} = 2^{16} = 65536
    \end{align*}

    \p{b}

    \p{i} Let \( B = \{(x, S) | x \in S, S \in 2^A \}\). Consider \( P \subset 2^A \), which
    contains all subsets of \( A = \{1, 2, 3, \ldots, n\} \) with cardinality 2:

    \[
        P = \{ \{1, 2\}, \{1,3\}, \ldots, \{1, n\}, \{2, 3\}, \{2, 4\}, \ldots, \{n-1, n\} \}
    \]

    The cardinality of \( P \) is \( \binom{n}{2} \), representing the number of ways to choose
    pairs from \( n \) elements. For an element of \( P \), such as \( S_1 = \{1,2\}\), two pairs
    are contributed to \( B \): \( (1, S_1) \) and \( (2, S_1) \). Similarly, each element of \( P
    \) contributes exactly two pairs to \( B \). Thus, with \( \binom{n}{2} \) elements in \( P \),
    the total contribution to \( B \) is \( 2 \binom{n}{2} \). \par

    Likewise, all subsets of \( A \) with three elements each, contribute \( 3 \binom{n}{3} \) pairs
    to \( B \), those with four elements contribute \( 4 \binom{n}{4} \) pairs, etc. Thus,
    generalizing the patter, the cardinality of \( B \) can be calculated as: 

    \begin{align} \label{card_B} 
        |B| = \sum_{x=0}^{n} x\binom{n}{x} 
    \end{align}

    Notice that the sum starts at 0, since it corresponds to the empty set. When \( S = \emptyset
    \), the statement \( x \in S \) is not valid, hence no pair is contributed to \( B \).

    Now consider the binomial theorem: 
    \begin{align*}
        (1 + a)^n = \sum_{x=0}^n a^x \binom{n}{x}
    \end{align*}

    Differentiating with respect to \( a \) gives: 
    \begin{align*}
        n(1+a)^{n-1} &= \sum_{x = 0}^n xa^{x-1} \binom{n}{x}
    \end{align*}

    Let \( a = 1 \), then:
    \begin{align*}
        n2^{n-1} &= \sum_{x = 0}^n x \binom{n}{x}
    \end{align*}

    The right hand side of the above equation is equal to \eqref{card_B}, hence the cardinality of \( B \)
    is:
    \[
        |B| = n2^{n-1}
    \] \q

    \end{ps}


\section*{5.}
\begin{ps}

    \p{i} For each of the two arguments in the domain, there are three choices of images in the
    range, therefore there are \( 3^2 = \mathbf{9} \) maps for the given sets.

    \p{ii} The are \textbf{6} injective maps for the given sets.
        \begin{center}
        \begin{tabular}{|c|c|c|}
        \hline
        Function & Image of 1 & Image of 2 \\
        \hline
        $f_1$ & 1 & 2 \\
        $f_2$ & 1 & 3 \\
        $f_3$ & 2 & 1 \\
        $f_4$ & 2 & 3 \\
        $f_5$ & 3 & 1 \\
        $f_6$ & 3 & 2 \\
        \hline
        \end{tabular}
        \end{center}

    \p{iii} There are \textbf{0} bijective maps for the given sets, since the number of elements in
    the domain does not match the number of elements in the range.

    \p{iv} Similar to (i), there are \( 2^3 = \textbf{8} \) maps for the given sets. 

    \p{v} There are \textbf{6} surjective maps for the given sets.

        \begin{center}
        \begin{tabular}{|c|c|c|c|}
        \hline
        Function & Image of 1 & Image of 2 & Image of 3 \\
        \hline
        $f_1$ & 1 & 2 & 1 \\
        $f_2$ & 1 & 2 & 2 \\
        $f_3$ & 2 & 1 & 1 \\
        $f_4$ & 2 & 1 & 2 \\
        $f_5$ & 1 & 1 & 2 \\
        $f_6$ & 2 & 2 & 1 \\
        \hline
        \end{tabular}
        \end{center}

\end{ps}

\section*{6.}
\section*{7.}
\section*{8.}

\end{document}
