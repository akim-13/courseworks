\documentclass[12pt]{article}

% Packages
\usepackage{amsmath, amsthm, amssymb, amsfonts}
\usepackage{mathtools}
\usepackage[hidelinks]{hyperref}
\usepackage{enumitem}
\usepackage{geometry}
\usepackage{fancyhdr}
\usepackage{cancel}

% Page Geometry
\geometry{a4paper, margin=1in, headheight=14.5pt}

% Header and Footer
\pagestyle{fancy}
\fancyhf{}
\lhead{Discrete Mathematics Problem Sheet}
\rhead{CM12004}
\cfoot{\thepage}

% Custom Commands
\newcommand{\p}[1]{\item[\textnormal{(#1)}]}
\newcommand{\q}{\hfill $\blacksquare$}
\newcommand{\nir}{\mathrel{\mathrlap{/}*}}

\newenvironment{ps}
{\begin{enumerate}[leftmargin=0em, itemindent=1.5em]}
{\end{enumerate}}

\begin{document}

\begin{center}
\begin{tabular}{|c|c|}
\hline
\textbf{Notation} & \textbf{Meaning}  \\
\hline

T & True  \\
F & False  \\
\(\oplus\) & XOR \\
\( \downarrow \) & NOR \\
\( \uparrow \) & NAND \\
\( \blacksquare \) & Indicates the end of proof/solution \\
\( \overset{d}{\equiv} \) & Equivalent by definition \\
\(\mathbb{E}\) & Set of all even integers \\
\(\mathbb{O}\) & Set of all odd integers \\
\( x*y \) & \( (x, y) \in R \) \\
\(x \nir y \) & \( (x, y) \notin R \) \\

\hline
\end{tabular}
\end{center}

\section*{1.}
\begin{ps}

    \p{i} The formula \(P \rightarrow Q\) is \textbf{neither} a tautology nor identically
    false.
        \begin{center}
        \begin{tabular}{|c|c|c|c|c|}
        \hline
        $X$ & $Y$ & $P = X \vee Y$ & $Q = \neg (X \wedge Y)$ & $P \rightarrow Q$ \\
        \hline
        T & T & T & F & F \\
        T & F & T & T & T \\
        F & T & T & T & T \\
        F & F & F & T & T \\
        \hline
        \end{tabular}
        \end{center}

    \p{ii} The formula \(P \rightarrow Q\) is \textbf{neither} a tautology nor identically
    false.

        \begin{center}
        \begin{tabular}{|c|c|c|c|c|}
        \hline
        $X$ & $Y$ & $P = X \vee Y$ & $Q = \neg X \wedge \neg Y$ & $P \rightarrow Q$ \\
        \hline
        T & T & T & F & F \\
        T & F & T & F & F \\
        F & T & T & F & F \\
        F & F & F & T & T \\
        \hline
        \end{tabular}
        \end{center}

    \p{iii} The formula \(P \rightarrow Q\) is a \textbf{tautology}. Note that the second
    part of Q (\(\neg X \vee X\)) is also a tautology.

        \begin{center}
        \begin{tabular}{|c|c|c|c|c|}
        \hline
        $X$ & $Y$ & $P = X \rightarrow Y$ & $Q = (\neg X \vee Y) \wedge (\neg X \vee X)$ & $P \rightarrow Q$ \\
        \hline
        T & T & T & T & T \\
        T & F & F & F & T \\
        F & T & T & T & T \\
        F & F & T & T & T \\
        \hline
        \end{tabular}
        \end{center}

    \p{iv} The formula \(P \rightarrow Q\) is a \textbf{tautology}.

        \begin{center}
        \begin{tabular}{|c|c|c|c|c|}
        \hline
        $X$ & $Y$ & $P = X \rightarrow \neg Y$ & $Q = Y \rightarrow \neg X$ & $P \rightarrow Q$ \\
        \hline
        T & T & F & F & T \\
        T & F & T & T & T \\
        F & T & T & T & T \\
        F & F & T & T & T \\
        \hline
        \end{tabular}
        \end{center}

    \p{v} The formula \(P \rightarrow Q\) is a \textbf{tautology}.

        \begin{center}
        \begin{tabular}{|c|c|c|c|c|c|}
        \hline
        $X$ & $Y$ & $Z$ & $P = X \wedge (Y \vee Z)$ & $Q = (X \vee Y) \wedge (X \vee Z)$ & $P \rightarrow Q$ \\
        \hline
        T & T & T & T & T & T \\
        T & T & F & T & T & T \\
        T & F & T & T & T & T \\
        T & F & F & F & T & T \\
        F & T & T & F & T & T \\
        F & T & F & F & F & T \\
        F & F & T & F & F & T \\
        F & F & F & F & F & T \\
        \hline
        \end{tabular}
        \end{center}

    \p{vi} The formula \(P \rightarrow Q\) is \textbf{neither} a tautology nor identically
    false.

        \begin{center}
        \begin{tabular}{|c|c|c|c|c|}
        \hline
        $X$ & $Y$ & $P = X \rightarrow Y$ & $Q = \neg X \rightarrow \neg Y$ & $P \rightarrow Q$ \\
        \hline
        T & T & T & T & T \\
        T & F & F & T & T \\
        F & T & T & F & F \\
        F & F & T & T & T \\
        \hline
        \end{tabular}
        \end{center}

    \p{vii} The formula \(P \rightarrow Q\) is \textbf{neither} a tautology nor
    identically false.

        \begin{center}
        \begin{tabular}{|c|c|c|c|c|}
        \hline
        $X$ & $Y$ & $P = X \rightarrow Y$ & $Q = \neg (Y \rightarrow X)$ & $P \rightarrow Q$ \\
        \hline
        T & T & T & F & F \\
        T & F & F & F & T \\
        F & T & T & T & T \\
        F & F & T & F & F \\
        \hline
        \end{tabular}
        \end{center}

    \p{viii} The formula \(P \rightarrow Q\) is a \textbf{tautology}.

        \begin{center}
        \begin{tabular}{|c|c|c|c|c|c|}
        \hline
        $X$ & $Y$ & $Z$ & $P = (Y \rightarrow Z) \wedge (X \rightarrow Y) $ & $Q = X \rightarrow Z$ & $P \rightarrow Q$ \\
        \hline
        T & T & T & T & F & T \\
        T & T & F & F & F & T \\
        T & F & T & F & T & T \\
        T & F & F & F & F & T \\
        F & T & T & T & T & T \\
        F & T & F & F & T & T \\
        F & F & T & T & T & T \\
        F & F & F & T & T & T \\
        \hline
        \end{tabular}
        \end{center}

\end{ps}


\section*{2.}
\begin{ps}

    \p{a} There exist \textbf{16} different binary logical connectives. There are
    4 possible combinations of T and F that a binary connective can take in. Each
    combination can result in 2 outcomes (T or F), therefore \(2^4 = 16\).

    \p{b} This section will make use of the following identities:

        \begin{enumerate}[label=\arabic*.]
            \item \( X \wedge Y \equiv \neg(\neg X \vee \neg Y) \)
            \item \( X \vee Y \equiv \neg(\neg X \wedge \neg Y) \)
            \item \( X \rightarrow Y \equiv \neg X \vee Y \)
        \end{enumerate}

    \p{i} Consider the definition of XOR: 
        \begin{equation}
            \label{XOR}
            X \oplus Y \equiv (X \vee Y) \wedge \neg(X \wedge Y)
        \end{equation}

        Let us transform equation \eqref{XOR} as follows:
        \begin{align*}
            (X \vee Y) \wedge \neg(X \wedge Y) &\equiv (X \vee Y) \wedge (\neg X \vee \neg Y) \\
                                               &\equiv \neg(\neg X \wedge \neg Y) \wedge \neg (X
                                               \wedge Y)
        \end{align*} \q

    \p{ii} Let us transform equation \eqref{XOR} as follows: 
        \begin{align}
            \label{b ii}
            (X \vee Y) \wedge \neg(X \wedge Y) &\equiv (X \vee Y) \wedge (\neg X \vee \neg Y) \notag \\
                                               &\equiv \neg (\neg (X \vee Y) \vee \neg (\neg X
                                               \vee Y ))
        \end{align} \q

    \p{iii} Let us transform equation \eqref{b ii} as follows: 
        \begin{align*}
            \neg (\neg (X \vee Y) \vee \neg (\neg X \vee Y )) &\equiv \neg(\neg(\neg X \rightarrow
            Y) \vee \neg(X \rightarrow \neg Y)) \\
            &\equiv \neg ((\neg X \rightarrow Y) \rightarrow \neg (X \rightarrow \neg Y))
        \end{align*} \q

    \p{c} By definition of NAND: 
        \[
            X \wedge Y \equiv \neg (X \uparrow Y)
        \]   

        Consider the truth table for NAND:

        \begin{center}
        \begin{tabular}{|c|c|c|}
        \hline
        $X$ & $Y$ & $X \uparrow Y$ \\
        \hline
        T & T & F \\
        T & F & T \\
        F & T & T \\
        F & F & T \\
        \hline
        \end{tabular}
        \end{center}

        Notice that \( X \uparrow X \equiv \neg X \). Hence: 
        \begin{align*}
            &\neg (X \uparrow Y) \equiv (X \uparrow Y) \uparrow (X \uparrow Y) \Leftrightarrow \\
            &X \wedge Y \equiv (X \uparrow Y) \uparrow (X \uparrow Y)
        \end{align*}
        Similarly, using the second and third identities, the remaining two connectives can be
        expressed as follows: 
        \begin{align*}
            X \vee Y &\equiv \neg (\neg X \wedge \neg Y) \\
                     &\equiv \neg ((X \uparrow X) \wedge (Y \uparrow Y)) \\
                     &\overset{d}{\equiv} (X \uparrow X) \uparrow (Y \uparrow Y) \\
            X \rightarrow Y &\equiv \neg X \vee Y \\
                            &\equiv \neg (X \wedge \neg Y) \\
                            &\overset{d}{\equiv} X \uparrow \neg Y \\
                            &\equiv X \uparrow (Y \uparrow Y)
        \end{align*} \q

    \p{d} By definition of NOR: 
        \[
            X \vee Y \equiv \neg (X \downarrow Y)
        \]   

        Consider the truth table for NOR:

        \begin{center}
        \begin{tabular}{|c|c|c|}
        \hline
        $X$ & $Y$ & $X \downarrow Y$ \\
        \hline
        T & T & F \\
        T & F & F \\
        F & T & F \\
        F & F & T \\
        \hline
        \end{tabular}
        \end{center}

        Notice that \( X \downarrow X \equiv \neg X \). Hence: 
        \begin{align*}
            &\neg (X \downarrow Y) \equiv (X \downarrow Y) \downarrow (X \downarrow Y) \Leftrightarrow \\
            &X \vee Y \equiv (X \downarrow Y) \downarrow (X \downarrow Y)
        \end{align*}

        Similarly, using the second and third identities, the remaining two connectives can be
        expressed as follows: 
        \begin{align*}
            X \wedge Y &\equiv \neg (\neg X \vee \neg Y) \\
                     &\overset{d}{\equiv} \neg X \downarrow \neg Y \\
                     &\equiv (X \downarrow X) \downarrow (Y \downarrow Y) \\
            X \rightarrow Y &\equiv \neg X \vee Y \\
                            &\equiv \neg \neg (\neg X \vee Y) \\
                            &\overset{d}{\equiv} \neg (\neg X \downarrow Y)\\
                            &\equiv ((X \downarrow X) \downarrow Y) \downarrow ((X \downarrow X)
                            \downarrow Y)
        \end{align*} \q

\end{ps}


\section*{3.}
\begin{ps}

    \p{a}
    \p{i} Let \( y = x^2 \), then the given predicate transforms into a tautology:
        \[ 
            x^2 < x^2 + 1 \implies 0< 1 
        \]
        Hence, the given statement is \textbf{true}.

    \p{ii} The given statement is \textbf{false}. Consider a counterexample. Let \( y = -1 \), then: 
    \[
        x^2 < -1 + 1 \implies x^2 < 0
    \]   
    Since no such integer \( x \) exists, the above statement is false.
    
    \p{iii} The given statement is \textbf{false}. Suppose such \( y \) exists. Let \( x = y + 2 \),
    then: 
    \begin{align*}
        (y + 2)^2 &< y + 1 \\
        y^2 + 4y + 4 &< y + 1 \\
        y^2 + 3y + 3 &< 0 \\
    \end{align*}
    However, the quadratic \( y^2 + 3y + 3 \) is always positive for all integer numbers \( y \),
    therefore the assumption leads to a contradiction, and no such \( y \) exists.

    \p{iv} Let \( y = 2x \), then:
    \begin{align*}
        (x < 2x) &\rightarrow (x^2 < 4x^2) \\
        (x > 0) &\rightarrow (x^2 > 0)
    \end{align*}
    Since the above statement is a tautology, the given statement is \textbf{true}.

    \p{b}
    \p{i} Let \( x = -1, y = -1 \), then: 
    \[ 
        (-1)^2 < -1 + 1 \Leftrightarrow 1 < 0 \text{ is false.}
    \]\q

    \p{ii} Let \( x = 2, y = 2 \), then:
    \[
        2^2 < 4 + 1 \Leftrightarrow 4 < 5 \text{ is true.} 
    \]\q

    \p{iii} Let \( y = -1 \), then:
    \[
        x^2 < -1 + 1 \Leftrightarrow x^2 < 0 \text{ is identically false.}
    \]\q

    \p{iv} Let \( x = 0 \), then:
    \[
        (y > 0) \rightarrow (y^2 > 0) \text{ is a tautology.}
    \]\q

\end{ps}


\section*{4.}

\begin{ps}
    \p{a} This section will make use of the following theorem: \( |2^A| = 2^{|A|} \).
    \begin{align*}
        & |2^\emptyset| = 2^{|\emptyset|} = 2^0 = 1 \\
        & |2^{\{0\}}| = 2^1 = 2 \\
        & |2^{\{0\} \cup \{1\}}| = |2^{\{0,1\}}| = 2^2 = 4 \\
        & |2^{\{0\} \cap \{1\}}| = |2^\emptyset| = 1 \\
        & |2^{\{\emptyset, 0, 1\}}| = 2^3 = 8 \\
        &\left|2^{2^{2^{\{0,1\}}}}\right| = 2^{\left|2^{2^{\{0, 1\}}}\right|} 
        = 2^{2^{\left| 2^{\{0, 1\}}\right|}} = 2^{2^4} = 2^{16} = 65536
    \end{align*}

    \p{b}

    \p{i} Let \( B = \{(x, S) | x \in S, S \in 2^A \}\). Consider \( P \subset 2^A \), which
    contains all subsets of \( A = \{1, 2, 3, \ldots, n\} \) with cardinality 2:

    \[
        P = \{ \{1, 2\}, \{1,3\}, \ldots, \{1, n\}, \{2, 3\}, \{2, 4\}, \ldots, \{n-1, n\} \}
    \]

    The cardinality of \( P \) is \( \binom{n}{2} \), representing the number of ways to choose
    pairs from \( n \) elements. For an element of \( P \), such as \( S_1 = \{1,2\}\), two pairs
    are contributed to \( B \): \( (1, S_1) \) and \( (2, S_1) \). Similarly, each element of \( P
    \) contributes exactly two pairs to \( B \). Thus, with \( \binom{n}{2} \) elements in \( P \),
    the total contribution to \( B \) is \( 2 \binom{n}{2} \). \par

    Likewise, all subsets of \( A \) with three elements each, contribute \( 3 \binom{n}{3} \) pairs
    to \( B \), those with four elements contribute \( 4 \binom{n}{4} \) pairs, etc. Thus,
    generalizing the patter, the cardinality of \( B \) can be calculated as: 

    \begin{align} \label{card_B} 
        |B| = \sum_{x=0}^{n} x\binom{n}{x} 
    \end{align}

    Notice that the sum starts at 0, since it corresponds to the empty set. When \( S = \emptyset
    \), the statement \( x \in S \) is not valid, hence no pair is contributed to \( B \).

    Now consider the binomial theorem: 
    \begin{align*}
        (1 + a)^n = \sum_{x=0}^n a^x \binom{n}{x}
    \end{align*}

    Differentiating with respect to \( a \) gives: 
    \begin{align*}
        n(1+a)^{n-1} &= \sum_{x = 0}^n xa^{x-1} \binom{n}{x}
    \end{align*}

    Let \( a = 1 \), then:
    \begin{align*}
        n2^{n-1} &= \sum_{x = 0}^n x \binom{n}{x}
    \end{align*}

    The right hand side of the above equation is equal to \eqref{card_B}, hence the cardinality of \( B \)
    is:
    \[
        |B| = n2^{n-1}
    \] \q

    
    \p{ii} Let \( B = \{(S, T) | S \in 2^A, T \in 2^A, S \cap T = \emptyset \} \). 

    Consider the smallest value of \( n = 2 \):
    \begin{align*}
        &A = \{1, 2\} \\
        &2^A = \{\emptyset, \{1\}, \{2\}, \{1, 2\} \} 
    \end{align*}
    
    Now consider all possible pairs \( (U, k) \), where \( U \in 2^A \) and \( k \in \mathbb{N} \)
    represents the number of distinct subsets in \( 2^A \) with which \( U \) can be paired, such
    that \( U \) and each of these subsets have an empty intersection:

    \begin{enumerate}[label=\arabic*.]
        \item \( (\emptyset, 4) \Leftrightarrow B = \{ (\emptyset, \emptyset), (\emptyset, \{1\}),
            (\emptyset, \{2\}), (\emptyset, \{1,2\}) \), \dots \}
        \item \( (\{1\}, 2) \Leftrightarrow B = \{ \dots ,(\{1\}, \emptyset), (\{1\}, \{2\}), \dots \} \)
        \item \( (\{2\}, 2) \Leftrightarrow B = \{ \dots ,(\{2\}, \emptyset), (\{2\}, \{1\}), \dots \} \)
        \item \( (\{1,2\}, 1) \Leftrightarrow B = \{ \dots ,(\{1,2\}, \emptyset) \} \)
    \end{enumerate}

    Let \( n = 3 \), then:
    \begin{align*}
        &A = \{1, 2, 3\} \\
        &2^A = \{\emptyset, \{1\}, \{2\}, \{3\}, \{1, 2\}, \{1,3\}, \{2, 3\}, \{1, 2, 3\} \} 
    \end{align*}

    The pairs \( (U, k) \) are:
    \begin{enumerate}[label=\arabic*.]
        \item \( (\emptyset, 8) \)
        \item \( (\{1\}, 4) \)
        \item \( (\{2\}, 4) \)
        \item \( (\{3\}, 4) \)
        \item \( (\{1,2\}, 2) \)
        \item \( (\{1,3\}, 2) \)
        \item \( (\{2,3\}, 2) \)
        \item \( (\{1,2,3\}, 1) \)
    \end{enumerate}

    Observe the emerging common pattern. Since each element can always form at least one pair, the
    total number of cases to be considered equals to \( |2^A| = 2^{|A|} \) (\( 2^2 = 4 \) when \( n
    = 2 \), and \( 2^3 = 8 \) when \( n = 3 \)). Furthermore, notice that \( k \) goes down by the
    factor of 2 with each extra element in \( U \). Lastly, there are \( \binom{n}{|U|} \) cases
    which are considered for each cardinality of \( U \).

    Thus, to calculate the cardinality of \( B \), we have to find the sum of the
    product of \( k \) (the number of pairs for a given cardinality of \( U \)) and the number of
    possible ways to choose \( |U| \) elements from the collection of \( n \) elements: 
    \[
        |B| = \sum_{|U|=0}^n 2^{|U|}\binom{n}{n - |U|}
    \]
    Let us denote \( |U| \) as \( x \), and simplify the above sum using the fact that the number of
    ways to choose \( i \) elements from \( j \) elements equals to the number of ways to exclude \(
    j - i \) elements from \( j \) elements:
    
    \begin{align}
        \label{sum2x}
        |B| = \sum_{x=0}^n 2^x \binom{n}{x}
    \end{align}
    
    Now consider the binomial theorem:
    \[
        (a+b)^n = \sum_{x=0}^n a^{n-x} b^x \binom{n}{x}
    \]   
    Let \( a = 1 \) and \( b = 2 \):
    
    \begin{align*}
        (1 + 2)^n &= \sum_{x=0}^n 1^{n-x}2^x \binom{n}{x}\\
        3^n &= \sum_{x=0}^n 2^x \binom{n}{x}
    \end{align*} 
    Hence \eqref{sum2x} equals to \( 3^n \), leading to the final answer:
    \[
        |B| = 3^n 
    \] \q
    
    \textit{Remark}. The result can be interpreted intuitively as being analogous to the power set.
    Another way to calculate the cardinality of \( B \) is to find the number of triples of
    pair-wise disjoint sets \( (S, T, A \backslash (S \cup T)) \), as indicated by the given hint. In this case,
    there are three options for each element of \( A \): it is either in \( T \), \( S \) or it is
    excluded. Since there are \( n \) elements and each one has three choices, the total number of
    such triples is \( 3 \times 3 \time 3 \times \dots \times 3 \) \( n \) times, i.e. \( 3^n \).

\end{ps}



\section*{5.}
\begin{ps}

    \p{i} For each of the two arguments in the domain, there are three choices of images in the
    range, therefore there are \( 3^2 = \mathbf{9} \) maps for the given sets.

    \p{ii} The are \textbf{6} injective maps for the given sets.
        \begin{center}
        \begin{tabular}{|c|c|c|}
        \hline
        Function & Image of 1 & Image of 2 \\
        \hline
        $f_1$ & 1 & 2 \\
        $f_2$ & 1 & 3 \\
        $f_3$ & 2 & 1 \\
        $f_4$ & 2 & 3 \\
        $f_5$ & 3 & 1 \\
        $f_6$ & 3 & 2 \\
        \hline
        \end{tabular}
        \end{center}

    \p{iii} There are \textbf{0} bijective maps for the given sets, since the number of elements in
    the domain does not match the number of elements in the range.

    \p{iv} Similar to (i), there are \( 2^3 = \textbf{8} \) maps for the given sets. 

    \p{v} There are \textbf{6} surjective maps for the given sets.

        \begin{center}
        \begin{tabular}{|c|c|c|c|}
        \hline
        Function & Image of 1 & Image of 2 & Image of 3 \\
        \hline
        $f_1$ & 1 & 2 & 1 \\
        $f_2$ & 1 & 2 & 2 \\
        $f_3$ & 2 & 1 & 1 \\
        $f_4$ & 2 & 1 & 2 \\
        $f_5$ & 1 & 1 & 2 \\
        $f_6$ & 2 & 2 & 1 \\
        \hline
        \end{tabular}
        \end{center}

\end{ps}

\section*{6.}

This section will make use of the following structure:
\[
    \text{Proposition } P(n)
\]   

\begin{enumerate}
    \item Base case.
    \item Inductive hypothesis.
    \item Inductive step.
    \item Conclusion.
\end{enumerate}

\begin{ps}

    \p{a} \[ P(n) = \sum_{1 \leq i \leq n} (2i - 1) = n^2, \forall n \in \mathbb{N} \]
    \begin{enumerate}[label=\arabic*.]
        \item Let \( n = 1 \), then: 
        \begin{align*}
            2 \times 1 - 1 &= 1^2 \\
            1 &= 1
        \end{align*}
        Hence, P(1) is true.

    \item Let us assume that \( P(k) \text{ is true, where } k \in \mathbb{N} \) is an arbitrary
        fixed number.

    \item Let n = k + 1, then:
    \begin{align*}
        \sum_{1 \leq i \leq k+1} (2i-1) &= (k+1)^2 \\
        \sum_{1 \leq i \leq k} (2i-1) + 2(k+1) - 1 &= (k+1)^2 
    \end{align*}
    Using the inductive hypothesis:
    \begin{align*}
            k^2 + 2k + 2 - 1 &= k^2 + 2k + 1 \\
            k^2 + 2k + 1 &= k^2 + 2k + 1
    \end{align*}
    Hence \( P(k+1) \) is true.

    \item The proposition \( P(k+1) \) has been proven to be true for some arbitrary \( k \in
        \mathbb{N} \) under the assumption that \( P(k) \) is true. Since \( P(1) \) has also been
        shown to be true, by the principles of mathematical induction, \( P(n) \) is true for all \( n \in
        \mathbb{N} \).

    \end{enumerate} \q

    \p{b} \[ P(n) = \sum_{1 \leq i \leq n} i^2 = \frac{1}{6}n(n+1)(2n+1), \forall n \in
    \mathbb{N} \]

    \begin{enumerate}[label=\arabic*.]
        \item Let \( n = 1 \), then: 
        \begin{align*}
            1^2 &= \frac{1}{6}(1)(1+1)(2(1) + 1) \\
            1 &= 1
        \end{align*}
        Hence, P(1) is true.

    \item Let us assume that \( P(k) \text{ is true, where } k \in \mathbb{N} \) is an arbitrary
        fixed number.

    \item Let n = k + 1, then:
    \begin{align*}
        \sum_{1 \leq i \leq k+1} i^2 &= \frac{1}{6}(k+1)(k+2)(2k+3) \\
        \sum_{1 \leq i \leq k} i^2 + (k+1)^2 &= \frac{1}{6}(2k^3 + 9k^2 + 13k + 6)
    \end{align*}
    Using the inductive hypothesis:
    \begin{align*}
        \frac{1}{6}k(k+1)(2k+1) + k^2 + 2k + 1 &= \frac{1}{3}k^3 + \frac{3}{2}k^2 + \frac{13}{6}k + 1 \\
         \frac{1}{3}k^3 + \frac{3}{2}k^2 + \frac{13}{6}k + 1 &= \frac{1}{3}k^3 + \frac{3}{2}k^2 + \frac{13}{6}k + 1
    \end{align*}
    Hence \( P(k+1) \) is true.

    \item The proposition \( P(k+1) \) has been proven to be true for some arbitrary \( k \in
        \mathbb{N} \) under the assumption that \( P(k) \) is true. Since \( P(1) \) has also been
        shown to be true, by the principles of mathematical induction, \( P(n) \) is true for all \( n \in
        \mathbb{N} \).

    \end{enumerate} \q
    
    \p{c}
    \begin{align*}
        \sum_{3 \leq i \leq n-2} i^2 =
        \sum_{1 \leq i \leq n-2} i^2 - \sum_{1 \leq i \leq 2} i^2 =
        \sum_{1 \leq i \leq n-2} i^2 - 5
    \end{align*}
    Using the result from part (b), the equation above transforms into:
    \begin{align*}
        &\frac{1}{6}(n-2)(n-1)(2n-3) - 5 = \\
        &\frac{1}{6}(2n^3 - 9n^2 + 13n - 6) - 5 = \\
        &\frac{1}{3}n^3 - \frac{3}{2}n^2 + \frac{13}{6}n - 6, \forall n \in \mathbb{N}, n \geq 5
    \end{align*} \q

    \p{d} 
    \[
        P(n): 7^n - 1 \text{ is divisible by 6 for all } n \in \mathbb{N}_0.
    \]   

    \begin{enumerate}[label=\arabic*.]
        \item Let \( n = 0 \), then: 
        \begin{align*}
            7^0 - 1 = 1 - 1 = 0
        \end{align*}
        Since 0 is divisible by 6, P(0) is true.

    \item Let us assume that \( P(k) \text{ is true, where } k \in \mathbb{N}_0 \) is an arbitrary
        fixed number. Therefore, \( P(k) \) can be expressed as \( P(k) = 6m \), where \( m \in \mathbb{N}_0 \).

    \item Consider \( P(k+1) - P(k) \):
    \begin{align*}
        7^{k+1} - 1 - 7^k + 1 &= \\
        7^k (7-1) &= \\
        6(7^k)
    \end{align*}
    Thus \( P(k+1) \) can be expressed as follows:
    \begin{align*}
        P(k+1) = 6(7^k) + P(k)
    \end{align*}
    Using the inductive hypothesis: 
    \begin{align*}
        P(k+1) = 6(7^k) + 6m = 6(7^k + m) \text{ is divisible by 6.}
    \end{align*}

    Hence \( P(k+1) \) is true.

    \item The proposition \( P(k+1) \) has been proven to be true for some arbitrary \( k \in
        \mathbb{N}_0 \) under the assumption that \( P(k) \) is true. Since \( P(0) \) has also been
        shown to be true, by the principles of mathematical induction, \( P(n) \) is true for all \( n \in
        \mathbb{N}_0 \).

    \end{enumerate} \q

    \p{e}
    \[
        P(n): 2n + 1 \leq 2^n, \forall n \in \mathbb{N}, n \geq 3
    \]   
    \begin{enumerate}[label=\arabic*.]
        \item Let \( n = 3 \), then: 
        \begin{align*}
            2(3) + 1 &\leq 2^3 \\
            7 &\leq 8
        \end{align*}
        Hence, P(3) is true.

    \item Let us assume that \( P(k) \text{ is true, where } k \in \mathbb{N} \) is an arbitrary
        fixed number.

    \item Consider \( P(k) \):
    \begin{align*}
        2k + 1 &\leq 2^k \quad \biggr| \times 2 \\
        4k + 2 &\leq 2^{k+1}
    \end{align*}
    Now consider the statement \( 2k+3 \leq 4k + 2 \):
    \begin{align*}
        2k + 3 &\leq 4k + 2 \\
        2k &\geq 1 \\
        k &\geq \frac{1}{2}
    \end{align*}
    Thus, the above statement is true for all \( k \in \mathbb{N}, k \geq 3 \).
    Therefore, using the principle of transitivity:
    \begin{align*}
        2k+3 &\leq 2^{k+1} \\
        2(k+1) + 1 &\leq 2^{k+1}
    \end{align*}
    Hence \( P(k+1) \) is true.

    \item The proposition \( P(k+1) \) has been proven to be true for some arbitrary \( k \in
        \mathbb{N} \) under the assumption that \( P(k) \) is true. Since \( P(3) \) has also been
        shown to be true, by the principles of mathematical induction, \( P(n) \) is true for all \( n \in
        \mathbb{N}, n \geq 3 \).

    \end{enumerate} \q

\end{ps}

\section*{7.}
This section will make use of the following structure:
\[
    \text{Relation } R \text{ on } \mathbb{Z}.
\]   
\begin{enumerate}

    \item Check if \( R \) is reflexive (for any \(x \in Z, x*x \)).
    \item Check if \( R \) is symmetric (if \( x*y \), then \( y*x \)).
    \item Check if \( R \) is transitive (if \( x*y \) and \( y*z \), then \( x*z \)).

\end{enumerate}

Furthermore, the following facts will be utilized:
\begin{enumerate}

    \item If \(a \in \mathbb{E} \text{ and } a = b + c \), then it must be the case that either both
        \( b,c \in \mathbb{E} \) or both \( b,c \in \mathbb{O} \).

    \item If \(a \in \mathbb{E} \text{ and } a = bc \), then it must be the case that either
        \( b \in \mathbb{E} \text{ or } c \in \mathbb{E} \).

\end{enumerate}


\begin{ps}

    \p{i}
        \[
            R = \{ x*y \,|\, (x+y) \in \mathbb{O} \}
        \]   

        \begin{enumerate}[label=\arabic*.]

            \item \(\forall x \in \mathbb{Z} (x + x = 2x \in \mathbb{E}) \implies x \nir x \implies
                R \) is \textbf{irreflexive}.

            \item If \( (x+y) \in \mathbb{O} \), then \( x+y = (y+x) \in \mathbb{O} \implies y*x
                \implies R \) is \textbf{symmetric}.

            % CLARIFY
            \item Suppose \( x \in \mathbb{E} \) and \( (x+y) \in \mathbb{O} \). Moreover, suppose
                \( z \in \mathbb{E} \) and \( (y+z) \in \mathbb{O} \). Then \( (x+z) \in \mathbb{E}
                \implies x \nir z \implies R \)  is \textbf{intransitive}.

        \end{enumerate}

    \p{ii}
        \[
            R = \{ x*y \,|\, (x+y) \in \mathbb{E} \}
        \]   

        \begin{enumerate}[label=\arabic*.]

            \item \(\forall x \in \mathbb{Z} (x + x = 2x \in \mathbb{E}) \implies x * x
                \implies R \) is \textbf{reflexive}.

            \item If \( (x+y) \in \mathbb{E} \), then \( x+y = (y+x) \in \mathbb{E} \implies
                y*x \implies R \) is \textbf{symmetric}.
            
            \item Consider the case when \( x, y \in \mathbb{O} \). Then \( (x+y) \in \mathbb{E} \),
                and for \( (y+z) \in \mathbb{E} \) to be true, \( z \in \mathbb{O} \) must be true.
                Therefore, \( (x+z) \in \mathbb{E} \implies x*z \).

                Now consider an alternative case when \( x,y \in \mathbb{E} \). Then \( (x+y) \in
                \mathbb{E} \), and for \( (y+z) \in \mathbb{E} \) to be true, \( z \in \mathbb{E} \)
                Therefore, \( (x+z) \in \mathbb{E} \implies x*z \).

                Since both cases have shown that \( x*z \), \( R \) is \textbf{transitive}.

        \end{enumerate}

    \p{iii}
        \[
            R = \{ x*y \,|\, xy \in \mathbb{O} \}
        \]   

        \begin{enumerate}[label=\arabic*.]

            \item Let \(x = 2 \text{, then } x^2 = 4 \in \mathbb{E} \implies x \nir x \implies
                R \) is \textbf{irreflexive}.

            \item If \( xy \in \mathbb{O} \), then \( xy = yx \in \mathbb{O} \implies y*x
                \implies R \) is \textbf{symmetric}.

            \item If \( xy, yz \in \mathbb{O} \), then \( x, y, z \in \mathbb{O} \) must be true.
                Therefore \( xz \in \mathbb{O} \implies x*z \implies R\) is \textbf{transitive}.
 
        \end{enumerate}
    %
    % \p{iv}
    %     \[
    %         R = \{ x*y \,|\, (x+xy) \in \mathbb{E} \}
    %     \]   
    %
    %     \begin{enumerate}[label=\arabic*.]
    %
    %         \item \(\forall x \in \mathbb{Z} (x + x^2 = 2x \in \mathbb{E}) \implies x * x
    %             \implies R \) is \textbf{reflexive}.
    %
    %         \item If \( (x+y) \in \mathbb{E} \), then \( x+y = (y+x) \in \mathbb{E} \implies
    %             y*x \implies R \) is \textbf{symmetric}.
    %         
    %         \item Consider the case when \( x, y \in \mathbb{O} \). Then \( (x+y) \in \mathbb{E} \),
    %             and for \( (y+z) \in \mathbb{E} \) to be true, \( z \in \mathbb{O} \) must be true.
    %             Therefore, \( (x+z) \in \mathbb{E} \implies x*z \).
    %
    %             Now consider an alternative case when \( x,y \in \mathbb{E} \). Then \( (x+y) \in
    %             \mathbb{E} \), and for \( (y+z) \in \mathbb{E} \) to be true, \( z \in \mathbb{E} \)
    %             Therefore, \( (x+z) \in \mathbb{E} \implies x*z \).
    %
    %             Since both cases have shown that \( x*z \), \( R \) is \textbf{transitive}.
    %
    %     \end{enumerate}

\end{ps}

\section*{8.}

\end{document}
